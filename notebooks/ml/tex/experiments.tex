\section{Experiments}

The following experiments refer to 3-fold cross-validation over \emph{linearly} and \emph{nonlinearly} separable generated datasets of size 100, so the reported results are to considered as a mean over the 3 folds.

\subsection{Support Vector Classifier}

Below experiments are about the SVC for which I tested different values for the regularization hyperparameter $C$, i.e., from \emph{soft} to \emph{hard margin}, and in case of nonlinearly separable data also different \emph{kernel functions} mentioned above.

\subsubsection{Hinge loss}

\paragraph{Primal formulation}

The experiments results shown in \ref{primal_svc_hinge_cv_results} referred to \emph{AdaGrad} algorithm are obtained with $\alpha$, i.e, the \emph{learning rate} or \emph{step size}, setted to 0.5. Training is stopped if after 5 iterations the training loss is not lower than the best found so far.

\begin{table}[H]
\centering
\caption{SVC Primal formulation results with Hinge loss}
\label{primal_svc_hinge_cv_results}
\begin{tabular}{lllrrrrrr}
\toprule
          &     &   &  fit\_time &  n\_iter &  train\_accuracy &  val\_accuracy &  train\_n\_sv &  val\_n\_sv \\
solver & C & momentum &           &         &                 &               &             &           \\
\midrule
sgd & 1   & none &  0.614727 &    2222 &        0.987525 &      0.985075 &          38 &        21 \\
          &     & standard &  0.386092 &    1417 &        0.987525 &      0.985075 &          36 &        20 \\
          &     & nesterov &  0.396583 &    1368 &        0.985019 &      0.980100 &          34 &        20 \\
          & 10  & none &  1.435118 &    4602 &        0.990012 &      0.980100 &          13 &         7 \\
          &     & standard &  1.195032 &    4068 &        0.987506 &      0.985075 &          12 &         7 \\
          &     & nesterov &  1.120775 &    3940 &        0.992500 &      0.985075 &          12 &         7 \\
          & 100 & none &  0.505901 &    1919 &        0.990012 &      0.985075 &           5 &         3 \\
          &     & standard &  0.292062 &     911 &        0.990012 &      0.985075 &           6 &         4 \\
          &     & nesterov &  0.286854 &    1060 &        0.990012 &      0.985075 &           5 &         3 \\
liblinear & 1   & - &  0.001517 &     146 &        0.979969 &      0.979798 &          11 &         5 \\
          & 10  & - &  0.001479 &     397 &        0.982475 &      0.984848 &           7 &         5 \\
          & 100 & - &  0.002547 &    1000 &        0.979950 &      0.984924 &           6 &         4 \\
\bottomrule
\end{tabular}
\end{table}


\paragraph{Linear Dual formulations}

The experiments results shown in \ref{linear_lagrangian_dual_svc_cv_results} are obtained with $\alpha$, i.e, the \emph{learning rate} or \emph{step size}, setted to 0.5 for the \emph{AdaGrad} algorithm.

\begin{tabular}{llrrrrr}
\toprule
optimizer &   C &  fit\_time &  train\_accuracy &  test\_accuracy &  nr\_train\_sv &  nr\_test\_sv \\
\midrule
   cvxopt &   1 &  0.046403 &        0.990012 &       0.990050 &           12 &          12 \\
   libsvm &   1 &  0.004200 &        0.990012 &       0.990050 &           26 &          26 \\
      smo &   1 &  0.071781 &        0.990012 &       0.990050 &           12 &          12 \\
   cvxopt &  10 &  0.025000 &        0.992519 &       0.980100 &            7 &           7 \\
   libsvm &  10 &  0.004469 &        0.997512 &       0.990050 &           13 &          13 \\
      smo &  10 &  0.085767 &        0.992519 &       0.980100 &            7 &           7 \\
   cvxopt & 100 &  0.020040 &        0.992519 &       0.980100 &            6 &           6 \\
   libsvm & 100 &  0.003470 &        0.997512 &       0.985075 &           10 &          10 \\
      smo & 100 &  0.121755 &        0.992519 &       0.980100 &            6 &           6 \\
\bottomrule
\end{tabular}


\begin{tabular}{llrrrrrr}
\toprule
  ld &   C &  fit\_time &  train\_accuracy &  val\_accuracy &  n\_iter &  train\_n\_sv &  val\_n\_sv \\
\midrule
bcqp &   1 &  0.005570 &        0.982531 &      0.980100 &       0 &         130 &       130 \\
  qp &   1 &  0.005950 &        0.979987 &      0.984999 &       0 &         131 &       131 \\
bcqp &  10 &  0.004206 &        0.982531 &      0.980100 &       0 &         130 &       130 \\
  qp &  10 &  0.005344 &        0.979987 &      0.984999 &       0 &         131 &       131 \\
bcqp & 100 &  0.004327 &        0.982531 &      0.980100 &       0 &         130 &       130 \\
  qp & 100 &  0.005249 &        0.979987 &      0.984999 &       0 &         131 &       131 \\
\bottomrule
\end{tabular}


\paragraph{Nonlinear Dual formulations}

The experiments results shown in \ref{nonlinear_dual_svc_cv_results} and \ref{nonlinear_lagrangian_dual_svc_cv_results} are obtained with \emph{d} and \emph{r} hyperparameters equal to 3 and 1 respectively for the \emph{polynomial} kernel; \emph{gamma} is setted to \emph{`scale`} for both \emph{polynomial} and \emph{gaussian RBF} kernels. Moreover, the experiments results shown in \ref{nonlinear_lagrangian_dual_svc_cv_results} are obtained with $\alpha$, i.e, the \emph{learning rate} or \emph{step size}, setted to 0.5 for the \emph{AdaGrad} algorithm.

\begin{table}[h!]
\centering
\caption{Nonlinear SVC Wolfe Dual formulation results with Hinge loss}
\label{nonlinear_dual_svc_cv_results}
\begin{tabular}{lllrlrrrr}
\toprule
    &     &     &  fit\_time & n\_iter &  train\_accuracy &  val\_accuracy &  train\_n\_sv &  val\_n\_sv \\
solver & kernel & C &           &        &                 &               &             &           \\
\midrule
cvxopt & poly & 1   &  0.083115 &      - &        0.841242 &      0.671062 &          31 &        31 \\
    &     & 10  &  0.070198 &      - &        0.961172 &      0.870609 &           9 &         9 \\
    &     & 100 &  0.070366 &      - &        0.986248 &      0.972562 &           8 &         8 \\
    & rbf & 1   &  0.085442 &      - &        0.998747 &      1.000000 &          48 &        48 \\
    &     & 10  &  0.068151 &      - &        0.998747 &      1.000000 &          16 &        16 \\
    &     & 100 &  0.072836 &      - &        0.997494 &      1.000000 &          13 &        13 \\
libsvm & poly & 1   &  0.002998 &    139 &        1.000000 &      1.000000 &          28 &        28 \\
    &     & 10  &  0.003866 &    330 &        1.000000 &      1.000000 &          10 &        10 \\
    &     & 100 &  0.002908 &    118 &        1.000000 &      1.000000 &           8 &         8 \\
    & rbf & 1   &  0.002391 &     93 &        1.000000 &      1.000000 &          42 &        42 \\
    &     & 10  &  0.004044 &    242 &        1.000000 &      1.000000 &          14 &        14 \\
    &     & 100 &  0.003652 &    240 &        1.000000 &      1.000000 &          12 &        12 \\
smo & poly & 1   &  0.384261 &    140 &        0.847489 &      0.673568 &          29 &        29 \\
    &     & 10  &  0.493594 &    266 &        0.958675 &      0.868103 &           9 &         9 \\
    &     & 100 &  0.417218 &    221 &        0.987497 &      0.975068 &           7 &         7 \\
    & rbf & 1   &  0.287050 &     42 &        0.998747 &      1.000000 &          45 &        45 \\
    &     & 10  &  0.263121 &     58 &        0.998747 &      1.000000 &          15 &        15 \\
    &     & 100 &  0.212068 &     87 &        0.998747 &      1.000000 &          11 &        11 \\
\bottomrule
\end{tabular}
\end{table}


\begin{tabular}{lllrrrrrr}
\toprule
  ld & kernel &   C &  fit\_time &  train\_accuracy &  val\_accuracy &  n\_iter &  train\_n\_sv &  val\_n\_sv \\
\midrule
bcqp &   poly &   1 &  0.885154 &        0.751255 &      0.503741 &     341 &         218 &       218 \\
  qp &   poly &   1 &  0.871670 &        0.750002 &      0.501253 &     170 &         187 &       187 \\
bcqp &    rbf &   1 &  0.043698 &        0.993734 &      0.952718 &       1 &         234 &       234 \\
  qp &    rbf &   1 &  1.263577 &        0.756245 &      0.506266 &     189 &         161 &       161 \\
bcqp &   poly &  10 &  0.865091 &        0.751255 &      0.503741 &     341 &         218 &       218 \\
  qp &   poly &  10 &  1.202677 &        0.750002 &      0.501253 &     170 &         187 &       187 \\
bcqp &    rbf &  10 &  0.073544 &        0.993734 &      0.952718 &       1 &         234 &       234 \\
  qp &    rbf &  10 &  1.550249 &        0.751251 &      0.503759 &     375 &         141 &       141 \\
bcqp &   poly & 100 &  1.031816 &        0.751255 &      0.503741 &     341 &         218 &       218 \\
  qp &   poly & 100 &  0.855477 &        0.750002 &      0.501253 &     170 &         187 &       187 \\
bcqp &    rbf & 100 &  0.031768 &        0.993734 &      0.952718 &       1 &         234 &       234 \\
  qp &    rbf & 100 &  1.426301 &        0.751255 &      0.503759 &     377 &         107 &       107 \\
\bottomrule
\end{tabular}



\subsubsection{Squared Hinge loss}

\paragraph{Primal formulation}

The experiments results shown in \ref{primal_svc_squared_hinge_cv_results} referred to \emph{Stochastic Gradient Descent} algorithm are obtained with $\alpha$, i.e, the \emph{learning rate} or \emph{step size}, setted to 0.001 and $\beta$, i.e, the \emph{momentum}, equal to 0.4. The batch size is equal to 20. Training is stopped if after 5 iterations the training loss is not lower than the best found so far.

\begin{table}[H]
\centering
\caption{SVC Primal formulation results with Squared Hinge loss}
\label{primal_svc_squared_hinge_cv_results}
\begin{tabular}{lllrrrrrr}
\toprule
          &     &   &  fit\_time &  n\_iter &  train\_accuracy &  val\_accuracy &  train\_n\_sv &  val\_n\_sv \\
solver & C & momentum &           &         &                 &               &             &           \\
\midrule
sgd & 1   & none &  0.313461 &    1273 &        0.944956 &      0.939846 &          40 &        20 \\
          &     & standard &  0.253786 &     930 &        0.947462 &      0.939846 &          37 &        19 \\
          &     & nesterov &  0.245869 &     934 &        0.947462 &      0.939846 &          37 &        19 \\
          & 10  & none &  0.117412 &     384 &        0.952456 &      0.944821 &          26 &        13 \\
          &     & standard &  0.078306 &     239 &        0.952456 &      0.944821 &          25 &        13 \\
          &     & nesterov &  0.075893 &     244 &        0.952456 &      0.944821 &          25 &        13 \\
          & 100 & none &  0.038147 &      78 &        0.952456 &      0.944821 &          18 &        10 \\
          &     & standard &  0.051053 &      83 &        0.959975 &      0.944821 &          15 &         8 \\
          &     & nesterov &  0.045448 &      78 &        0.957468 &      0.944821 &          16 &         8 \\
liblinear & 1   & - &  0.001865 &     600 &        0.964968 &      0.954847 &          29 &        14 \\
          & 10  & - &  0.002564 &    1000 &        0.962462 &      0.954847 &          28 &        14 \\
          & 100 & - &  0.002419 &    1000 &        0.959956 &      0.934946 &          27 &        15 \\
\bottomrule
\end{tabular}
\end{table}




\subsection{Support Vector Regression}

Below experiments are about the SVR for which I tested different values for regularization hyperparameter $C$, i.e., from \emph{soft} to \emph{hard margin}, the $\epsilon$ penalty value and in case of nonlinearly separable data also different \emph{kernel functions} mentioned above.

\subsubsection{Epsilon-insensitive loss}

\paragraph{Primal formulation}

The experiments results shown in \ref{primal_svr_eps_cv_results} referred to \emph{AdaGrad} algorithm are obtained with $\alpha$, i.e, the \emph{learning rate} or \emph{step size}, setted to 0.5. Training is stopped if after 5 iterations the training loss is not lower than the best found so far.

\begin{table}[h!]
\centering
\caption{SVR Primal formulation results with Epsilon-insensitive loss}
\label{primal_svr_eps_cv_results}
\begin{tabular}{lllrrrrrr}
\toprule
          &     &     &  fit\_time &  n\_iter &  train\_r2 &    val\_r2 &  train\_n\_sv &  val\_n\_sv \\
solver & C & epsilon &           &         &           &           &             &           \\
\midrule
adagrad & 1   & 0.1 &  0.483018 &     873 &  0.919206 &  0.915684 &          66 &        33 \\
          &     & 0.2 &  0.501837 &     897 &  0.919990 &  0.916504 &          66 &        33 \\
          &     & 0.3 &  0.498535 &     880 &  0.920085 &  0.916655 &          65 &        33 \\
          & 10  & 0.1 &  2.049033 &    3542 &  0.977834 &  0.972868 &          65 &        32 \\
          &     & 0.2 &  1.913584 &    3511 &  0.977801 &  0.972839 &          65 &        32 \\
          &     & 0.3 &  1.882657 &    3478 &  0.977783 &  0.972878 &          65 &        32 \\
          & 100 & 0.1 &  2.263548 &    4000 &  0.978120 &  0.974239 &          66 &        32 \\
          &     & 0.2 &  2.248653 &    4000 &  0.978118 &  0.974263 &          66 &        32 \\
          &     & 0.3 &  1.818380 &    4000 &  0.978120 &  0.974189 &          66 &        32 \\
liblinear & 1   & 0.1 &  0.000683 &      14 &  0.918827 &  0.916841 &          66 &        33 \\
          &     & 0.2 &  0.000565 &      12 &  0.918820 &  0.916672 &          65 &        32 \\
          &     & 0.3 &  0.000755 &      11 &  0.919212 &  0.916977 &          65 &        32 \\
          & 10  & 0.1 &  0.000760 &     103 &  0.977852 &  0.972051 &          65 &        33 \\
          &     & 0.2 &  0.000651 &     188 &  0.977844 &  0.971971 &          65 &        33 \\
          &     & 0.3 &  0.000603 &     105 &  0.977865 &  0.972111 &          64 &        33 \\
          & 100 & 0.1 &  0.000908 &     719 &  0.977723 &  0.974270 &          66 &        33 \\
          &     & 0.2 &  0.001058 &     689 &  0.977628 &  0.973889 &          65 &        33 \\
          &     & 0.3 &  0.001008 &     807 &  0.977658 &  0.974038 &          65 &        33 \\
\bottomrule
\end{tabular}
\end{table}


\paragraph{Linear Dual formulations}

The experiments results shown in \ref{linear_lagrangian_dual_svr_cv_results} are obtained with $\alpha$, i.e, the \emph{learning rate} or \emph{step size}, setted to 0.5 for the \emph{AdaGrad} algorithm.

\begin{table}[H]
\centering
\caption{Linear SVR Wolfe Dual formulation results with Epsilon-insensitive loss}
\label{linear_dual_svr_cv_results}
\begin{tabular}{lllrrrrrr}
\toprule
       &     &     &  fit\_time &  n\_iter &  train\_r2 &    val\_r2 &  train\_n\_sv &  val\_n\_sv \\
solver & C & epsilon &           &         &           &           &             &           \\
\midrule
smo & 10  & 0.2 &  0.139534 &     219 &  0.977926 &  0.972457 &          65 &        65 \\
       & 100 & 0.3 &  0.383196 &     900 &  0.977737 &  0.973939 &          66 &        66 \\
       &     & 0.1 &  0.524643 &    1508 &  0.977788 &  0.974139 &          66 &        66 \\
       & 10  & 0.3 &  0.053870 &      38 &  0.977953 &  0.972544 &          65 &        65 \\
       &     & 0.1 &  0.053832 &      56 &  0.977920 &  0.972445 &          66 &        66 \\
       & 1   & 0.3 &  0.039421 &      60 &  0.918942 &  0.915576 &          66 &        66 \\
       &     & 0.2 &  0.022597 &      13 &  0.918341 &  0.915019 &          66 &        66 \\
       &     & 0.1 &  0.017338 &      15 &  0.917773 &  0.914442 &          66 &        66 \\
       & 100 & 0.2 &  0.266404 &     394 &  0.977742 &  0.974022 &          66 &        66 \\
libsvm & 1   & 0.3 &  0.002088 &      54 &  0.918786 &  0.916554 &          66 &        66 \\
       & 10  & 0.1 &  0.001896 &     282 &  0.977852 &  0.972051 &          66 &        66 \\
       &     & 0.2 &  0.001990 &     193 &  0.977851 &  0.972025 &          65 &        65 \\
       &     & 0.3 &  0.001814 &     593 &  0.977870 &  0.972135 &          65 &        65 \\
       & 1   & 0.1 &  0.003532 &      63 &  0.917627 &  0.915448 &          66 &        66 \\
       & 100 & 0.1 &  0.003821 &    2621 &  0.977723 &  0.974270 &          66 &        66 \\
       &     & 0.2 &  0.003551 &    2709 &  0.977673 &  0.974122 &          66 &        66 \\
       & 1   & 0.2 &  0.003926 &     102 &  0.918194 &  0.915985 &          66 &        66 \\
       & 100 & 0.3 &  0.003090 &    4141 &  0.977655 &  0.974045 &          66 &        66 \\
cvxopt & 1   & 0.2 &  0.017439 &       9 &  0.918341 &  0.915058 &          67 &        67 \\
       & 100 & 0.3 &  0.021550 &       9 &  0.977737 &  0.973956 &          67 &        67 \\
       &     & 0.2 &  0.024708 &       9 &  0.977742 &  0.974033 &          67 &        67 \\
       &     & 0.1 &  0.020310 &       9 &  0.977788 &  0.974150 &          67 &        67 \\
       & 10  & 0.3 &  0.025372 &      10 &  0.977954 &  0.972562 &          66 &        66 \\
       &     & 0.2 &  0.018264 &       9 &  0.977926 &  0.972474 &          67 &        67 \\
       &     & 0.1 &  0.024514 &       9 &  0.977920 &  0.972466 &          67 &        67 \\
       & 1   & 0.3 &  0.013357 &      10 &  0.918942 &  0.915614 &          66 &        66 \\
       &     & 0.1 &  0.017656 &       9 &  0.917772 &  0.914479 &          67 &        67 \\
\bottomrule
\end{tabular}
\end{table}


\begin{table}[H]
\centering
\caption{Linear SVR Lagrangian Dual formulation results with Epsilon-insensitive loss}
\label{linear_lagrangian_dual_svr_cv_results}
\begin{tabular}{lllrrrr}
\toprule
     &     &     &  fit\_time &        r2 &  n\_iter &  n\_sv \\
dual & C & epsilon &           &           &         &       \\
\midrule
qp & 1   & 0.1 &  2.834686 &  0.731400 &    1000 &   100 \\
     &     & 0.2 &  2.834862 &  0.731400 &    1000 &   100 \\
     &     & 0.3 &  2.812738 &  0.731400 &    1000 &   100 \\
     & 10  & 0.1 &  2.840206 &  0.731400 &    1000 &   100 \\
     &     & 0.2 &  2.368673 &  0.731400 &    1000 &   100 \\
     &     & 0.3 &  2.633349 &  0.731400 &    1000 &   100 \\
     & 100 & 0.1 &  2.593171 &  0.731400 &    1000 &   100 \\
     &     & 0.2 &  2.531446 &  0.731400 &    1000 &   100 \\
     &     & 0.3 &  1.191494 &  0.731400 &    1000 &   100 \\
bcqp & 1   & 0.1 &  2.962745 &  0.733183 &    1000 &   100 \\
     &     & 0.2 &  2.746601 &  0.733183 &    1000 &   100 \\
     &     & 0.3 &  3.205020 &  0.733183 &    1000 &   100 \\
     & 10  & 0.1 &  3.041657 &  0.733183 &    1000 &   100 \\
     &     & 0.2 &  2.699528 &  0.733183 &    1000 &   100 \\
     &     & 0.3 &  2.801045 &  0.733183 &    1000 &   100 \\
     & 100 & 0.1 &  2.534525 &  0.733183 &    1000 &   100 \\
     &     & 0.2 &  2.534465 &  0.733183 &    1000 &   100 \\
     &     & 0.3 &  1.243881 &  0.733183 &    1000 &   100 \\
\bottomrule
\end{tabular}
\end{table}


\paragraph{Nonlinear Dual formulations}

The experiments results shown in \ref{nonlinear_dual_svr_cv_results} and \ref{nonlinear_lagrangian_dual_svr_cv_results} are obtained with \emph{d} and \emph{r} hyperparameters both equal to 3 for the \emph{polynomial} kernel; \emph{gamma} is setted to \emph{`scale`} for both \emph{polynomial} and \emph{gaussian RBF} kernels. Moreover, the experiments results shown in \ref{nonlinear_lagrangian_dual_svc_cv_results} are obtained with $\alpha$, i.e, the \emph{learning rate} or \emph{step size}, setted to 0.5 for the \emph{AdaGrad} algorithm.

\begin{table}[h!]
\centering
\caption{Nonlinear SVR Wolfe Dual formulation results with Epsilon-insensitive loss}
\label{nonlinear_dual_svr_cv_results}
\begin{tabular}{llllrrrlrr}
\toprule
       &     &     &     &     fit\_time &   train\_r2 &     val\_r2 &    n\_iter &  train\_n\_sv &  val\_n\_sv \\
solver & kernel & C & epsilon &              &            &            &           &             &           \\
\midrule
cvxopt & poly & 1   & 0.1 &     0.041113 &   0.905433 & -15.210301 &         1 &          30 &        30 \\
       &     &     & 0.2 &     0.020089 & -24.954851 &  -9.710897 &         1 &           5 &         5 \\
       &     &     & 0.3 &     0.011838 &  -0.215852 &  -7.639445 &         1 &           4 &         4 \\
       &     & 10  & 0.1 &     0.010001 &   0.811604 & -10.451673 &         1 &          31 &        31 \\
       &     &     & 0.2 &     0.013686 &  -2.558647 &  -8.951845 &         1 &           4 &         4 \\
       &     &     & 0.3 &     0.015321 &   0.445454 &  -7.324711 &         1 &           3 &         3 \\
       &     & 100 & 0.1 &     0.010776 &   0.897682 & -11.713805 &         1 &          51 &        51 \\
       &     &     & 0.2 &     0.013330 &  -2.558605 &  -8.951984 &         1 &           4 &         4 \\
       &     &     & 0.3 &     0.015465 &   0.445417 &  -7.324471 &         1 &           3 &         3 \\
       & rbf & 1   & 0.1 &     0.014610 &   0.983128 &   0.472774 &         - &          12 &        12 \\
       &     &     & 0.2 &     0.011270 &   0.965921 &  -0.685438 &         - &           7 &         7 \\
       &     &     & 0.3 &     0.012746 &   0.886375 &  -1.866503 &         - &           5 &         5 \\
       &     & 10  & 0.1 &     0.010773 &   0.987400 &   0.814540 &         - &           9 &         9 \\
       &     &     & 0.2 &     0.011876 &   0.964815 &  -0.687669 &         - &           6 &         6 \\
       &     &     & 0.3 &     0.011370 &   0.874593 &  -2.001040 &         - &           4 &         4 \\
       &     & 100 & 0.1 &     0.010134 &   0.981179 &   0.854367 &         - &           9 &         9 \\
       &     &     & 0.2 &     0.011038 &   0.962024 &  -0.630939 &         - &           6 &         6 \\
       &     &     & 0.3 &     0.011310 &   0.893251 &  -0.801456 &         - &           5 &         5 \\
smo & poly & 1   & 0.1 &    48.441721 &   0.851002 & -13.573915 &     63892 &          28 &        28 \\
       &     &     & 0.2 &     2.087687 & -24.226843 & -17.509435 &      2656 &           6 &         6 \\
       &     &     & 0.3 &     1.013869 &  -1.075241 & -13.457557 &      1340 &           4 &         4 \\
       &     & 10  & 0.1 &   321.763062 &   0.829192 & -10.919496 &    638453 &          28 &        28 \\
       &     &     & 0.2 &     3.081885 &  -1.965039 & -16.772837 &      3985 &           4 &         4 \\
       &     &     & 0.3 &     2.166342 &  -0.416636 & -13.148526 &      2726 &           3 &         3 \\
       &     & 100 & 0.1 &  2375.492913 &   0.888191 & -12.849364 &   5679480 &          28 &        28 \\
       &     &     & 0.2 &     3.046712 &  -1.965039 & -16.772837 &      3985 &           4 &         4 \\
       &     &     & 0.3 &     2.011438 &  -0.416636 & -13.148526 &      2726 &           3 &         3 \\
       & rbf & 1   & 0.1 &     0.042733 &   0.977543 &   0.447446 &        35 &          10 &        10 \\
       &     &     & 0.2 &     0.018814 &   0.961709 &  -0.706292 &        18 &           6 &         6 \\
       &     &     & 0.3 &     0.016197 &   0.877846 &  -2.174799 &        18 &           5 &         5 \\
       &     & 10  & 0.1 &     0.145227 &   0.984622 &   0.831171 &       168 &           8 &         8 \\
       &     &     & 0.2 &     0.019114 &   0.960406 &  -0.708836 &        22 &           6 &         6 \\
       &     &     & 0.3 &     0.012624 &   0.846434 &  -2.206505 &        16 &           4 &         4 \\
       &     & 100 & 0.1 &     0.161255 &   0.982501 &   0.835906 &       223 &           8 &         8 \\
       &     &     & 0.2 &     0.016944 &   0.960406 &  -0.708836 &        22 &           6 &         6 \\
       &     &     & 0.3 &     0.011717 &   0.846434 &  -2.206505 &        16 &           4 &         4 \\
libsvm & poly & 1   & 0.1 &     0.071900 &   0.978286 & -11.848929 &    220996 &          20 &        20 \\
       &     &     & 0.2 &     0.008648 &   0.970950 & -10.792492 &      5830 &           5 &         5 \\
       &     &     & 0.3 &     0.005080 &   0.919241 & -31.298810 &      2957 &           4 &         4 \\
       &     & 10  & 0.1 &     0.367158 &   0.977816 & -12.116107 &   1149782 &          20 &        20 \\
       &     &     & 0.2 &     0.016228 &   0.972071 & -10.791561 &      6665 &           4 &         4 \\
       &     &     & 0.3 &     0.003214 &   0.921527 & -31.296613 &      4236 &           4 &         4 \\
       &     & 100 & 0.1 &     5.444041 &   0.944878 &  -3.393675 &  35310042 &          28 &        28 \\
       &     &     & 0.2 &     0.003119 &   0.972071 & -10.791561 &      6665 &           4 &         4 \\
       &     &     & 0.3 &     0.022383 &   0.921527 & -31.296613 &      4236 &           4 &         4 \\
       & rbf & 1   & 0.1 &     0.016349 &   0.982884 &  -0.165508 &        96 &          18 &        18 \\
       &     &     & 0.2 &     0.004455 &   0.966819 &  -0.342140 &        24 &           6 &         6 \\
       &     &     & 0.3 &     0.007898 &   0.915276 &  -0.739398 &        11 &           5 &         5 \\
       &     & 10  & 0.1 &     0.007063 &   0.983896 &   0.533980 &       418 &          18 &        18 \\
       &     &     & 0.2 &     0.006231 &   0.967504 &  -0.342386 &        26 &           6 &         6 \\
       &     &     & 0.3 &     0.001741 &   0.923754 &  -0.734560 &        11 &           4 &         4 \\
       &     & 100 & 0.1 &     0.003601 &   0.984122 &   0.710024 &      3500 &          19 &        19 \\
       &     &     & 0.2 &     0.008419 &   0.967504 &  -0.342386 &        26 &           6 &         6 \\
       &     &     & 0.3 &     0.004023 &   0.923754 &  -0.734560 &        11 &           4 &         4 \\
\bottomrule
\end{tabular}
\end{table}


\begin{table}[h!]
\centering
\caption{Nonlinear SVR Lagrangian Dual formulation results with Epsilon-insensitive loss}
\label{nonlinear_lagrangian_dual_svr_cv_results}
\begin{tabular}{llllrrrrrr}
\toprule
   &     &     &     &  fit\_time &      train\_r2 &        val\_r2 &  n\_iter &  train\_n\_sv &  val\_n\_sv \\
dual & kernel & C & epsilon &           &               &               &         &             &           \\
\midrule
bcqp & poly & 1   & 0.1 &  0.081454 &  5.144076e-01 & -9.076995e+00 &      35 &          67 &        67 \\
   &     &     & 0.2 &  0.060330 &  5.079650e-01 & -5.349260e+00 &      40 &          67 &        67 \\
   &     &     & 0.3 &  0.088997 &  4.488241e-01 & -4.556107e+00 &      64 &          67 &        67 \\
   &     & 10  & 0.1 &  0.056643 &  5.144076e-01 & -9.076995e+00 &      35 &          67 &        67 \\
   &     &     & 0.2 &  0.076490 &  5.079650e-01 & -5.349260e+00 &      40 &          67 &        67 \\
   &     &     & 0.3 &  0.097213 &  4.488241e-01 & -4.556107e+00 &      64 &          67 &        67 \\
   &     & 100 & 0.1 &  0.060653 &  5.144076e-01 & -9.076995e+00 &      35 &          67 &        67 \\
   &     &     & 0.2 &  0.063831 &  5.079650e-01 & -5.349260e+00 &      40 &          67 &        67 \\
   &     &     & 0.3 &  0.105566 &  4.488241e-01 & -4.556107e+00 &      64 &          67 &        67 \\
   & rbf & 1   & 0.1 &  0.077817 &  7.396010e-01 & -1.390908e+00 &      33 &          67 &        67 \\
   &     &     & 0.2 &  0.185861 &  7.395830e-01 & -1.392041e+00 &      90 &          67 &        67 \\
   &     &     & 0.3 &  0.466749 &  5.929126e-01 & -2.757236e+00 &     196 &          67 &        67 \\
   &     & 10  & 0.1 &  0.108501 &  7.396010e-01 & -1.390908e+00 &      33 &          67 &        67 \\
   &     &     & 0.2 &  0.193361 &  7.395830e-01 & -1.392041e+00 &      90 &          67 &        67 \\
   &     &     & 0.3 &  0.457909 &  5.929126e-01 & -2.757236e+00 &     196 &          67 &        67 \\
   &     & 100 & 0.1 &  0.076684 &  7.396010e-01 & -1.390908e+00 &      33 &          67 &        67 \\
   &     &     & 0.2 &  0.211798 &  7.395830e-01 & -1.392041e+00 &      90 &          67 &        67 \\
   &     &     & 0.3 &  0.254640 &  5.929126e-01 & -2.757236e+00 &     196 &          67 &        67 \\
qp & poly & 1   & 0.1 &  0.135663 &  4.113756e-01 & -1.020445e+01 &      63 &          66 &        66 \\
   &     &     & 0.2 &  0.528343 &  3.492985e-01 & -7.300395e+00 &     347 &          64 &        64 \\
   &     &     & 0.3 &  0.549913 & -1.552200e+15 & -1.787602e+12 &     380 &          46 &        46 \\
   &     & 10  & 0.1 &  0.102671 &  4.113756e-01 & -1.020445e+01 &      63 &          66 &        66 \\
   &     &     & 0.2 &  0.550925 &  3.492985e-01 & -7.300395e+00 &     347 &          64 &        64 \\
   &     &     & 0.3 &  0.569163 & -1.552200e+15 & -1.787602e+12 &     380 &          46 &        46 \\
   &     & 100 & 0.1 &  0.098125 &  4.113756e-01 & -1.020445e+01 &      63 &          66 &        66 \\
   &     &     & 0.2 &  0.407858 &  3.492985e-01 & -7.300395e+00 &     347 &          64 &        64 \\
   &     &     & 0.3 &  0.401141 & -1.552200e+15 & -1.787602e+12 &     380 &          46 &        46 \\
   & rbf & 1   & 0.1 &  0.338488 &  6.913982e-01 & -1.635557e+00 &     132 &          67 &        67 \\
   &     &     & 0.2 &  0.442192 &  6.866212e-01 & -1.660805e+00 &     184 &          67 &        67 \\
   &     &     & 0.3 &  0.636889 &  6.115361e-01 & -2.297081e+00 &     257 &          67 &        67 \\
   &     & 10  & 0.1 &  0.128231 &  7.148036e-01 & -1.440764e+00 &      54 &          67 &        67 \\
   &     &     & 0.2 &  0.276627 &  7.066835e-01 & -1.470430e+00 &     106 &          67 &        67 \\
   &     &     & 0.3 &  0.336929 &  6.456839e-01 & -1.981865e+00 &     137 &          67 &        67 \\
   &     & 100 & 0.1 &  0.177903 &  7.148036e-01 & -1.440764e+00 &      54 &          67 &        67 \\
   &     &     & 0.2 &  0.272468 &  7.066835e-01 & -1.470430e+00 &     106 &          67 &        67 \\
   &     &     & 0.3 &  0.305527 &  6.456839e-01 & -1.981865e+00 &     137 &          67 &        67 \\
\bottomrule
\end{tabular}
\end{table}



\subsubsection{Squared Epsilon-insensitive loss}

\paragraph{Primal formulation}

The experiments results shown in \ref{primal_svr_squared_eps_cv_results} referred to \emph{Stochastic Gradient Descent} algorithm are obtained with $\alpha$, i.e, the \emph{learning rate} or \emph{step size}, setted to 0.001 and $\beta$, i.e, the \emph{momentum}, equal to 0.4. The batch size is equal to 20. Training is stopped if after 5 iterations the training loss is not lower than the best found so far.

\begin{table}[H]
\centering
\caption{SVR Primal formulation results with Squared Epsilon-insensitive loss}
\label{primal_svr_squared_eps_cv_results}
\begin{tabular}{llllrrrr}
\toprule
          &   &     &     &  fit\_time &        r2 &  n\_iter &  n\_sv \\
solver & momentum & C & epsilon &           &           &         &       \\
\midrule
sgd & none & 1   & 0.1 &  1.203146 &  0.977025 &     641 &   100 \\
          &   &     & 0.2 &  1.090908 &  0.977021 &     633 &    99 \\
          &   &     & 0.3 &  1.121058 &  0.977016 &     625 &    99 \\
          &   & 10  & 0.1 &  0.423047 &  0.977573 &      74 &   100 \\
          &   &     & 0.2 &  0.342571 &  0.977572 &      70 &    99 \\
          &   &     & 0.3 &  0.355486 &  0.977572 &      72 &    99 \\
          &   & 100 & 0.1 &  0.111561 &  0.977409 &       9 &   100 \\
          &   &     & 0.2 &  0.101309 &  0.977408 &       9 &    99 \\
          &   &     & 0.3 &  0.039628 &  0.977407 &       9 &    98 \\
          & standard & 1   & 0.1 &  0.710437 &  0.977035 &     398 &   100 \\
          &   &     & 0.2 &  0.586380 &  0.977031 &     392 &    99 \\
          &   &     & 0.3 &  0.694706 &  0.977025 &     385 &    99 \\
          &   & 10  & 0.1 &  0.097984 &  0.977572 &      42 &    99 \\
          &   &     & 0.2 &  0.159697 &  0.977571 &      40 &    99 \\
          &   &     & 0.3 &  0.151594 &  0.977572 &      42 &    99 \\
          &   & 100 & 0.1 &  0.017708 &  0.977439 &       7 &    99 \\
          &   &     & 0.2 &  0.022605 &  0.977438 &       7 &    99 \\
          &   &     & 0.3 &  0.033693 &  0.977441 &       7 &    98 \\
          & nesterov & 1   & 0.1 &  0.528662 &  0.977035 &     399 &   100 \\
          &   &     & 0.2 &  0.499102 &  0.977030 &     392 &    99 \\
          &   &     & 0.3 &  0.507848 &  0.977026 &     386 &    99 \\
          &   & 10  & 0.1 &  0.086664 &  0.977572 &      43 &    99 \\
          &   &     & 0.2 &  0.070886 &  0.977572 &      42 &    99 \\
          &   &     & 0.3 &  0.086196 &  0.977570 &      40 &    99 \\
          &   & 100 & 0.1 &  0.016199 &  0.977411 &       7 &   100 \\
          &   &     & 0.2 &  0.018226 &  0.977411 &       7 &    99 \\
          &   &     & 0.3 &  0.016476 &  0.977413 &       7 &    98 \\
liblinear & - & 1   & 0.1 &  0.001553 &  0.977554 &      96 &   100 \\
          &   &     & 0.2 &  0.001414 &  0.977553 &      96 &   100 \\
          &   &     & 0.3 &  0.001962 &  0.977551 &      96 &   100 \\
          &   & 10  & 0.1 &  0.006422 &  0.977577 &     826 &   100 \\
          &   &     & 0.2 &  0.007993 &  0.977576 &     826 &    99 \\
          &   &     & 0.3 &  0.005684 &  0.977576 &     839 &    99 \\
          &   & 100 & 0.1 &  0.008910 &  0.977538 &    1000 &   100 \\
          &   &     & 0.2 &  0.008264 &  0.977540 &    1000 &    99 \\
          &   &     & 0.3 &  0.007509 &  0.977541 &    1000 &    98 \\
\bottomrule
\end{tabular}
\end{table}
